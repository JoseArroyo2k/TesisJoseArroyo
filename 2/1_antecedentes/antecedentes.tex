En esta sección se presentarán diversos artículos de investigación o tesis las cuales abordarán diversas técnicas y enfoques que se emplearon para afrontar problemas similares al de esta tesis. Asimismo, a continuación se presenta un cuadro resumen (véase Anexo \ref{A:table}) de lo que se presenta en esta sección.


\subsection{Copper price estimation using bat algorithm \citep*{pr_dehghani2018copper}}
\citeauthor{pr_dehghani2018copper} realizaron un artículo de investigación el cual fue publicado en la revista «Resources Policy» en el año 2018. Este fue titulado \citetitle{pr_dehghani2018copper} la cual traducida al español significa «Estimación del precio del cobre utilizando el algoritmo bat».

\subsubsection{Planteamiento del Problema y objetivo }

La retinopatía diabética es una complicación grave de la diabetes, que puede conducir a la ceguera si no se detecta y trata a tiempo. La prevalencia de esta enfermedad y la necesidad de diagnósticos precisos y tempranos son cruciales para prevenir el deterioro irreversible de la visión. Los métodos tradicionales de detección, aunque efectivos, pueden ser lentos y están sujetos a la variabilidad de la interpretación humana. Esto ha llevado a la exploración del uso de modelos de aprendizaje profundo para mejorar la precisión y eficiencia del diagnóstico.

\subsubsection{Técnicas empleadas por los autores}

Deep Learning en Medicina: El aprendizaje profundo ha revolucionado el análisis de imágenes médicas, proporcionando herramientas para mejorar significativamente la detección y clasificación de diversas enfermedades, incluida la retinopatía diabética.
Redes Neuronales Convolutivas (CNN): Son particularmente útiles en el reconocimiento de patrones visuales complejos en imágenes médicas, como las que se encuentran en el examen del fondo del ojo en pacientes diabéticos.
Transferencia de Aprendizaje: Utilizando redes preentrenadas, los investigadores pueden aprovechar el conocimiento aprendido en grandes bases de datos de imágenes generales para mejorar la precisión en tareas específicas como la clasificación de la severidad de la retinopatía diabética.

\subsubsection{Metodología empleada por los autores}

Modelos y Datasets Utilizados: Se analizan modelos como VGG16 y DenseNet aplicados a datasets específicos como APTOS 2019 y Messidor, que contienen imágenes etiquetadas de retinopatía diabética.
Evaluación de Modelos: Se utilizan métricas de rendimiento como la precisión, sensibilidad y especificidad para evaluar la efectividad de los modelos de aprendizaje profundo en la identificación correcta de las etapas de la retinopatía.
Técnicas de Preprocesamiento: Incluyen la normalización y aumento de datos para mejorar la calidad y aumentar la variabilidad de las imágenes de entrenamiento, crucial para el entrenamiento eficaz de modelos de deep learning.

\subsubsection{Resultados obtenidos}

Avances en la Detección Automatizada: Los estudios demuestran que los modelos de deep learning pueden igualar o superar la precisión de los métodos de detección tradicionales, ofreciendo diagnósticos más rápidos y reduciendo la carga de trabajo manual.
Impacto Clínico: La implementación de estos sistemas de diagnóstico automatizado puede significar un avance significativo en la atención preventiva, potencialmente reduciendo la incidencia de ceguera entre los pacientes diabéticos.
Futuras Direcciones: Los resultados alentadores sugieren que la investigación futura podría centrarse en la integración de estos modelos en los sistemas de salud reales para pruebas clínicas a gran escala y evaluación de la aceptación por parte de los profesionales médicos.

 