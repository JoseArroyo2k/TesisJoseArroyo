En esta sección se presentarán diversos artículos de investigación o tesis que abordan diversas técnicas y enfoques utilizados para afrontar problemas similares al de esta tesis.

\subsection{Deep Convolutional Neural Networks for Detecting COVID-19 Using Medical Images: A Survey} %{\citep*{pr_dehghani2018copper}}
%\citeauthor{pr_dehghani2018copper} realizaron un artículo de investigación el cual fue publicado en la revista «Resources Policy» en el año 2018. Este fue titulado \citetitle{pr_dehghani2018copper} la cual traducida al español significa «Estimación del precio del cobre utilizando el algoritmo bat».

\subsubsection{Planteamiento del Problema y objetivo }

El COVID-19, causado por el virus SARS-CoV-2, se propagó rápidamente desde diciembre de 2019, representando una grave amenaza para la salud pública. La detección temprana y precisa es crucial para un tratamiento efectivo. Los métodos tradicionales tienen limitaciones en sensibilidad y tiempo, por lo que se investiga el uso de deep learning aplicado a imágenes médicas para mejorar la precisión y velocidad de diagnóstico.

\subsubsection{Técnicas empleadas por los autores}

Deep Learning en Medicina: Uso de redes neuronales convolutivas (CNN) para analizar imágenes médicas.
Redes Neuronales Convolutivas (CNN): Uso de arquitecturas como VGG, ResNet y DenseNet.
Transferencia de Aprendizaje: Redes preentrenadas en grandes bases de datos para mejorar la clasificación específica de COVID-19.

\subsubsection{Metodología empleada por lo	s autores}

Modelos y Datasets Utilizados: Evaluación de CNNs (VGG16, DenseNet, ResNet) aplicadas a datasets como COVIDx.
Evaluación de Modelos: Uso de métricas como precisión, sensibilidad, especificidad y AUC.
Técnicas de Preprocesamiento: Normalización y aumento de datos para mejorar la calidad y variabilidad de las imágenes.

\subsubsection{Resultados obtenidos}

Avances en la Detección Automatizada: Los modelos de deep learning pueden igualar o superar la precisión de los métodos tradicionales, ofreciendo diagnósticos más rápidos.
Impacto Clínico: La implementación de estos sistemas automatizados puede mejorar la detección temprana y reducir la carga de trabajo manual.
Futuras Direcciones: Integración de estos modelos en sistemas de salud reales para pruebas clínicas a gran escala.

\subsection{Heart Disease Detection Using Machine Learning and Deep Learning}

\subsubsection{Planteamiento del Problema y objetivo}

Las enfermedades del corazón se han convertido en una de las principales causas de muerte en todo el mundo, con aproximadamente 17.9 millones de muertes registradas anualmente. Detectar la presencia de enfermedades del corazón de manera temprana es crucial para monitorear y tratar a los pacientes a tiempo y así salvar vidas. Esta investigación tiene como objetivo utilizar técnicas de aprendizaje automático y profundo para detectar enfermedades cardíacas, mejorando la precisión y la eficacia del diagnóstico.

\subsubsection{Técnicas empleadas por los autores}

Deep Learning y Machine Learning en Medicina: Uso de redes neuronales y modelos de aprendizaje automático para analizar datos médicos.
Redes Neuronales Convolutivas (CNN) y Modelos de Aprendizaje Automático: Se emplean modelos como SVM, Logistic Regression, Decision Tree, FFNN y LSTM.
Selección de Características: Métodos como la matriz de correlación y el puntaje de Fisher para eliminar características no relevantes.

\subsubsection{Metodología empleada por los autores}

Modelos y Datasets Utilizados: Evaluación de modelos como Logistic Regression, SVM, Decision Tree, FFNN y LSTM aplicados a un conjunto de datos con 1026 registros.
Evaluación de Modelos: Uso de técnicas de validación cruzada y métricas como precisión, sensibilidad, especificidad y AUC.
Técnicas de Preprocesamiento: Métodos de normalización y selección de características para mejorar la calidad de los datos.

\subsubsection{Resultados obtenidos}

Avances en la Detección Automatizada: Los modelos de aprendizaje profundo y automático pueden igualar o superar la precisión de los métodos tradicionales, ofreciendo diagnósticos más rápidos y precisos.
Impacto Clínico: La implementación de estos sistemas automatizados puede mejorar la detección temprana y reducir la carga de trabajo manual en entornos clínicos.
Futuras Direcciones: Integración de estos modelos en sistemas de salud reales para pruebas clínicas a gran escala y evaluación de su eficacia en diversos contextos clínicos.

\subsection{Monitoring and Recognition of Heart Health using Heartbeat Classification with Deep Learning and IoT}

\subsubsection{Planteamiento del Problema y objetivo}

Las enfermedades cardiovasculares, como la arritmia y el infarto de miocardio, representan graves problemas de salud que pueden ser mortales si no se detectan y tratan a tiempo. La detección temprana y precisa es crucial para un tratamiento efectivo. El objetivo de este estudio es utilizar técnicas de aprendizaje profundo y el Internet de las cosas (IoT) para mejorar la detección y clasificación de latidos cardíacos, lo que permite identificar problemas de salud cardíaca de manera automática y precisa.

\subsubsection{Técnicas empleadas por los autores}

Deep Learning en Medicina: Uso de redes neuronales convolutivas (CNN) y técnicas de IoT para monitorear y analizar la salud cardíaca.
Redes Neuronales Convolutivas (CNN): Implementación de arquitecturas de CNN para la clasificación de latidos cardíacos.
Optimización de Algoritmos: Uso de algoritmos de optimización para mejorar la precisión y eficiencia de los modelos de clasificación.

\subsubsection{Metodología empleada por los autores}

Modelos y Datasets Utilizados: Evaluación de CNNs aplicadas a datasets como el MIT-BIH arrhythmia database.
Evaluación de Modelos: Uso de métricas como precisión, sensibilidad, especificidad y AUC.
Técnicas de Preprocesamiento: Segmentación y extracción de características de las señales ECG para mejorar la calidad y variabilidad de los datos.

\subsubsection{Resultados obtenidos}

Avances en la Detección Automatizada: Los modelos de deep learning, integrados con IoT, pueden igualar o superar la precisión de los métodos tradicionales, ofreciendo diagnósticos más rápidos y precisos.
Impacto Clínico: La implementación de estos sistemas automatizados puede mejorar la detección temprana de enfermedades cardíacas y reducir la carga de trabajo manual en entornos clínicos.
Futuras Direcciones: Integración de estos modelos en sistemas de salud reales para pruebas clínicas a gran escala y evaluación de su eficacia en diversos contextos clínicos.


\subsection{Advances in Deep Learning: From Diagnosis to Treatment}

\subsubsection{Planteamiento del Problema y objetivo}

El aprendizaje profundo ha revolucionado el campo del diagnóstico y tratamiento médico, alcanzando niveles de precisión comparables a los de los profesionales médicos en diversas tareas diagnósticas. Este estudio investiga el uso de modelos de aprendizaje profundo para integrar diferentes formas de datos médicos y proporcionar sugerencias diagnósticas y de tratamiento precisas y en tiempo real. El objetivo es mejorar la colaboración entre médicos y máquinas para ofrecer una atención médica personalizada y eficiente.

\subsubsection{Técnicas empleadas por los autores}

Deep Learning en Medicina**: Utilización de redes neuronales convolutivas (CNN) y modelos de aprendizaje profundo para tareas de diagnóstico y tratamiento.
Modelos de Fundamento Médico**: Empleo de modelos avanzados de deep learning que integran grandes conjuntos de datos y múltiples formas de información médica.
Optimización de Algoritmos**: Uso de técnicas de optimización para mejorar la precisión y eficiencia de los modelos en diversas tareas médicas.

\subsubsection{Metodología empleada por los autores}

Modelos y Datasets Utilizados**: Evaluación de modelos como R-CNN, U-Net y GPT-4 aplicados a conjuntos de datos médicos variados.
Evaluación de Modelos**: Uso de métricas como precisión, sensibilidad, especificidad y AUC para evaluar la efectividad de los modelos.
Técnicas de Preprocesamiento**: Normalización y aumento de datos para mejorar la calidad y variabilidad de los datos.

\subsubsection{Resultados obtenidos}

Avances en la Detección Automatizada**: Los modelos de deep learning pueden igualar o superar la precisión de los métodos tradicionales, ofreciendo diagnósticos y sugerencias de tratamiento más rápidos y precisos.
Impacto Clínico**: La implementación de estos sistemas automatizados puede mejorar la detección temprana de enfermedades y reducir la carga de trabajo manual en entornos clínicos.
Futuras Direcciones**: Integración de estos modelos en sistemas de salud reales para pruebas clínicas a gran escala y evaluación de su eficacia en diversos contextos clínicos.

\subsection{A Study on Scope of Artificial Intelligence in Diagnostic Medicine}

\subsubsection{Planteamiento del Problema y objetivo}

La inteligencia artificial (IA) ha demostrado un potencial significativo para mejorar el diagnóstico médico al analizar grandes volúmenes de datos de manera rápida y precisa. Este estudio analiza cómo las técnicas de IA, incluyendo el aprendizaje automático y profundo, pueden integrarse en la medicina diagnóstica para mejorar la precisión del diagnóstico y optimizar la atención al paciente. El objetivo es explorar el alcance y las aplicaciones de la IA en el diagnóstico médico para mejorar los resultados de los pacientes y reducir los costos de atención médica.

\subsubsection{Técnicas empleadas por los autores}

Inteligencia Artificial en Medicina: Uso de algoritmos de aprendizaje automático y profundo para analizar datos médicos.
Reconocimiento de Imágenes: Empleo de IA para examinar imágenes médicas como rayos X, tomografías computarizadas (CT) y resonancias magnéticas (MRI).
Procesamiento del Lenguaje Natural (NLP): Análisis de registros de salud electrónicos (EHR) y patrones de lenguaje para identificar indicios de enfermedades.

\subsubsection{Metodología empleada por los autores}

Modelos y Datasets Utilizados: Evaluación de algoritmos de IA aplicados a conjuntos de datos médicos diversos.
Evaluación de Modelos: Uso de métricas como precisión, sensibilidad, especificidad y AUC para evaluar la efectividad de los modelos.
Técnicas de Preprocesamiento: Normalización y aumento de datos para mejorar la calidad y variabilidad de los datos médicos.

\subsubsection{Resultados obtenidos}

Avances en la Detección Automatizada: Los modelos de IA pueden igualar o superar la precisión de los métodos tradicionales, ofreciendo diagnósticos más rápidos y precisos.
Impacto Clínico: La implementación de sistemas automatizados de IA puede mejorar la detección temprana de enfermedades y reducir la carga de trabajo manual en entornos clínicos.
Futuras Direcciones: Integración de modelos de IA en sistemas de salud reales para pruebas clínicas a gran escala y evaluación de su eficacia en diversos contextos clínicos.

 