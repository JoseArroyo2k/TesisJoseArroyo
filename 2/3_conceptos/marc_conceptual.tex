\subsection{Inteligencia Artificial (IA)}

La inteligencia artificial (IA) es una disciplina de las ciencias computacionales dedicada a desarrollar sistemas que pueden realizar tareas que típicamente requieren inteligencia humana. Esto incluye el reconocimiento de voz, toma de decisiones, y análisis de datos, entre otros. Los sistemas de IA son capaces de aprender y adaptarse a través de algoritmos que procesan grandes volúmenes de datos, mejorando su desempeño con el tiempo.

\subsection{Sistemas de Diagnóstico Asistido por Computadora (CAD)}

Los sistemas CAD utilizan algoritmos de inteligencia artificial para asistir a los médicos en el diagnóstico de enfermedades. En oftalmología, estos sistemas han demostrado ser altamente efectivos en la detección temprana de la retinopatía diabética, proporcionando análisis automáticos rápidos y precisos, y ayudando a los médicos a confirmar diagnósticos y tomar decisiones informadas. Los beneficios incluyen:

\begin{itemize}
 \item \textbf{Análisis Automático:} Evaluación rápida y precisa de imágenes retinianas.
 \item \textbf{Segunda Opinión:} Ayuda a los médicos a confirmar diagnósticos.
 \item \textbf{Monitorización Continua:} Seguimiento de la progresión de la enfermedad.
\end{itemize}

\subsection{Base de Datos APTOS y Messidor}

Estas bases de datos contienen imágenes de fondo de ojo etiquetadas con diferentes grados de retinopatía diabética, utilizadas para entrenar y evaluar modelos de deep learning en la detección de esta enfermedad. Son recursos cruciales para la investigación y desarrollo de sistemas de diagnóstico asistido por computadora.

\subsection{Implementación Técnica}

\subsubsection{Selección de la Plataforma de Desarrollo}

Para la implementación de modelos de deep learning en la detección de retinopatía diabética, las plataformas recomendadas incluyen:

\begin{itemize}
    \item \textbf{TensorFlow:} Una biblioteca open-source de Google para el desarrollo y entrenamiento de modelos de machine learning y deep learning.
    \item \textbf{Keras:} Una API de alto nivel para construir y entrenar modelos de deep learning, que funciona sobre TensorFlow.
    \item \textbf{PyTorch:} Una biblioteca de machine learning desarrollada por Facebook, popular por su flexibilidad y facilidad de uso.
\end{itemize}

\subsubsection{Diseño de la Arquitectura del Modelo}

\begin{itemize}
    \item \textbf{Frontend:} No aplica en este contexto, ya que se enfoca en la construcción del modelo de deep learning.
    \item \textbf{Backend:} El servidor que aloja el modelo de deep learning, procesa las imágenes retinianas y genera diagnósticos.
    \item \textbf{Integraciones:} Con sistemas de información médica y bases de datos de imágenes para obtener y almacenar datos de entrenamiento y validación.
\end{itemize}

\subsubsection{Desarrollo del Modelo}

\begin{itemize}
    \item \textbf{Entrenamiento del Modelo de Deep Learning:} Utilizar datos de imágenes retinianas etiquetadas para entrenar el modelo, asegurando que pueda detectar diferentes grados de retinopatía diabética.
    \item \textbf{Desarrollo de la Base de Conocimientos:} Compilar y estructurar la información médica relevante y las imágenes etiquetadas en una base de datos accesible por el modelo.
    \item \textbf{Pruebas y Validación:} Realizar pruebas exhaustivas para asegurar la precisión y fiabilidad del modelo, incluyendo pruebas con datos no vistos previamente para evaluar la capacidad de generalización.
\end{itemize}
