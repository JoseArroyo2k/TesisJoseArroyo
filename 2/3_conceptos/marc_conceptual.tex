

\subsection{Inteligencia Artificial (IA)}
La inteligencia artificial es una rama de las ciencias computacionales que busca emular la inteligencia humana a través de algoritmos y máquinas, permitiendo realizar actividades de manera más eficiente que el ser humano.

\subsection{Machine Learning (Aprendizaje Automático)}
Machine learning es una subdisciplina de la inteligencia artificial que utiliza técnicas de ingeniería y estadística para desarrollar algoritmos que permiten a las máquinas aprender de datos y realizar predicciones o decisiones sin ser explícitamente programadas para cada tarea.

\subsection{Deep Learning (Aprendizaje Profundo)}
El deep learning es una rama del machine learning que utiliza redes neuronales artificiales con múltiples capas (redes profundas) para modelar patrones complejos en grandes volúmenes de datos.

\subsection{Redes Neuronales Convolutivas (CNNs)}
Las redes neuronales convolutivas son un tipo de red neuronal profunda que es especialmente eficaz para el procesamiento y análisis de imágenes debido a su capacidad para capturar características espaciales jerárquicas.

\subsection{Transferencia de Aprendizaje}
La transferencia de aprendizaje es una técnica en deep learning donde un modelo preentrenado en una gran base de datos se adapta (fine-tuning) para realizar tareas específicas con un conjunto de datos diferente, mejorando la eficiencia y precisión del modelo en nuevas tareas.

\subsection{Retinopatía Diabética}
La retinopatía diabética es una complicación de la diabetes que afecta los vasos sanguíneos de la retina, llevando a una disminución de la visión y, en casos severos, a la ceguera.

\subsection{Sistemas de Diagnóstico Asistido por Computadora (CAD)}
Los sistemas CAD son herramientas que utilizan algoritmos de inteligencia artificial para ayudar a los médicos en el diagnóstico de enfermedades, mejorando la precisión y rapidez de los diagnósticos.

\subsection{Preprocesamiento de Datos}
El preprocesamiento de datos incluye técnicas como la normalización, estandarización, y aumento de datos (data augmentation) para preparar los datos antes de entrenar los modelos de aprendizaje automático y profundo.

\subsection{Métricas de Evaluación}
Las métricas de evaluación son criterios utilizados para medir el rendimiento de los modelos de machine learning y deep learning. Incluyen precisión, sensibilidad, especificidad, AUC, y F1-score.

\subsection{Base de Datos APTOS y Messidor}
Estas bases de datos contienen imágenes de fondo de ojo etiquetadas con diferentes grados de retinopatía diabética, utilizadas para entrenar y evaluar modelos de deep learning en la detección de esta enfermedad.
