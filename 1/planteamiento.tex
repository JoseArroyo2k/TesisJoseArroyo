\chapter{PLANTEAMIENTO DEL PROBLEMA}
\section{Descripción de la Realidad Problemática}

La retinopatía diabética (RD) es una complicación grave de la diabetes y la principal causa de ceguera en adultos en edad laboral en países industrializados. Afecta de manera significativa la calidad de vida debido a la pérdida visual y representa un desafío económico considerable para los sistemas de salud debido a los altos costos de los tratamientos necesarios.

A nivel mundial, la diabetes afecta a más de 537 millones de adultos y se proyecta que esta cifra aumentará a más de 780 millones para 2045. La prevalencia de la RD es alarmantemente alta, con un estudio de meta-análisis indicando que aproximadamente el 34.6% de los diabéticos desarrollarán alguna forma de esta condición, y un 7% sufrirá de formas más severas como la retinopatía diabética proliferativa.

El impacto socioeconómico de la RD es considerable, ya que además de los costos de tratamiento, afecta la capacidad de trabajo de los individuos, contribuyendo a la pérdida de independencia y posibles estados depresivos. Los tratamientos avanzados como las inyecciones intraoculares y las cirugías de retina representan una carga financiera adicional para los pacientes y los sistemas de salud.

Los factores de riesgo clave incluyen la duración de la diabetes, un control glucémico inadecuado, y la hipertensión. La detección temprana y el tratamiento adecuado son cruciales para prevenir la progresión de la RD, sin embargo, muchos pacientes no reciben un diagnóstico ni tratamiento temprano debido a la ausencia de síntomas en las etapas iniciales.
\section{Formulación del Problema}

La retinopatía diabética representa un desafío significativo en el campo de la salud pública debido a su prevalencia creciente y su impacto severo en la calidad de vida de los pacientes. A medida que las tasas de diabetes continúan elevándose globalmente, la incidencia de complicaciones oculares graves también aumenta, poniendo en riesgo la visión de millones. Sin embargo, los sistemas de salud a menudo enfrentan dificultades para implementar estrategias efectivas de detección y tratamiento debido a limitaciones tanto en recursos como en la cobertura de los servicios de salud. Estos desafíos subrayan la necesidad de abordar el problema desde múltiples ángulos, incluyendo la mejora en la detección temprana, el acceso a tratamientos innovadores, y la educación de pacientes y proveedores de salud sobre la gestión efectiva de la diabetes y sus complicaciones.

\subsection{Problema General}
\newcommand{\ProblemaGeneral}{
	La inadecuada detección y tratamiento de la retinopatía diabética, especialmente en etapas tempranas, resulta en una alta incidencia de ceguera y deterioro visual severo entre la población diabética.
}
\ProblemaGeneral
\subsection{Problemas Espec\'{i}ficos}
\newcommand{\Pbone}{
	Deficiencia en los métodos de cribado que resulta en diagnósticos tardíos de RD.
}
\newcommand{\Pbtwo}{
	Acceso limitado a tratamientos efectivos y avanzados para pacientes con etapas avanzadas de RD.
}
\newcommand{\Pbthree}{
	Insuficiente educación y gestión de la diabetes entre los pacientes, lo que aumenta el riesgode desarrollar complicaciones severas de la RD.
}
\newcommand{\Pbfour}{
	W
}
\newcommand{\Pbfive}{
	ES
}

\begin{itemize}
	\item \Pbone
	\item \Pbtwo
	\item \Pbthree
\end{itemize}

\section{Objetivos de la Investigación}

La investigación sobre la retinopatía diabética busca abordar las carencias en el manejo actual de esta complicación diabética, con el fin de reducir su impacto en los pacientes afectados y en el sistema de salud. Este estudio se enfoca en desarrollar estrategias efectivas que puedan ser implementadas a nivel local y global para mejorar la prevención, detección, y tratamiento de esta condición.

\subsection{Objetivo General}
\newcommand{\ObjetivoGeneral}{
	Desarrollar e implementar un programa multidisciplinario para mejorar la detección y el manejo de la retinopatía diabética, con el fin de reducir la incidencia de ceguera y mejorar la calidad de vida en pacientes diabéticos.
}
\ObjetivoGeneral
\subsection{Objetivos Espec\'{i}ficos}
\newcommand{\Objone}{
	Implementar un sistema de cribado más eficaz y accesible para la detección temprana de la retinopatía diabética en comunidades de alto riesgo.
}
\newcommand{\Objtwo}{
	Desarrollar y evaluar la efectividad de nuevos tratamientos farmacológicos y tecnológicos para pacientes con RD en diferentes etapas de la enfermedad.
}
\newcommand{\Objthree}{
	Establecer programas educativos y de apoyo para mejorar la gestión de la diabetes y la prevención de la RD, dirigidos tanto a pacientes como a profesionales de la salud.

}
\newcommand{\Objfour}{
	hhhg
}
\newcommand{\Objfive}{
	ghhhg
}

\begin{itemize}
	\item {\Objone}
	\item {\Objtwo}
	\item {\Objthree}

\end{itemize}

\section{Justificación de la Investigación}

\subsection{Teórica}
Esta investigación se realiza 

\subsection{Práctica}
Al culminar la investigación 

\subsection{Metodológica}. 

\section{Delimitación del Estudio}

\subsection{Espacial}
Para la presente investigación 

\subsection{Temporal}
Los datos que serán necesari. 

\subsection{Conceptual}
Esta investigación se 

\section{Hipótesis}

\subsection{Hipótesis General}
\newcommand{\HipotesisGeneral}{
	El uso de técnicas de.
}
\HipotesisGeneral
\subsection{Hipótesis Específicas}
\newcommand{\Hone}{
	x
}
\newcommand{\Htwo}{
	y
}
\newcommand{\Hthree}{
	z	
}
\newcommand{\Hfour}{
	cv
}
\newcommand{\Hfive}{
	xws
}
\begin{itemize}
	\item \Hone
	\item \Htwo
	\item \Hthree
	\item \Hfour
	\item \Hfive
\end{itemize}

\subsection{Matriz de Consistencia}
A continuación se presenta la matriz de consistencia elaborada para la presente investigación (véase Anexo \ref{1:table}).
