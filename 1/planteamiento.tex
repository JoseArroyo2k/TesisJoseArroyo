\chapter{PLANTEAMIENTO DEL PROBLEMA}
\section{Descripción de la Realidad Problemática}

La retinopatía diabética (RD) constituye una complicación severa derivada de la diabetes, siendo la principal causa de pérdida de visión en adultos trabajadores en naciones desarrolladas. Esta afección impacta profundamente en la vida de las personas, no solo reduciendo su calidad de vida por la disminución visual, sino también imponiendo retos económicos significativos para los sistemas de salud debido al elevado precio de los tratamientos.

Globalmente, más de 537 millones de adultos viven con diabetes, y se estima que este número ascenderá a más de 780 millones para el año 2045. Un análisis exhaustivo muestra que alrededor del 34.6\% de estas personas desarrollarán alguna forma de RD, y un 7\% experimentará variantes severas como la retinopatía diabética proliferativa.

La repercusión socioeconómica de la RD es vasta, incluyendo no solo los costes directos de los tratamientos sino también la reducción de la capacidad laboral de los afectados, lo que puede llevar a la pérdida de autonomía y al desarrollo de trastornos emocionales. Las opciones de tratamiento avanzadas, como las inyecciones intraoculares y las intervenciones quirúrgicas retinianas, suponen un lastre económico adicional tanto para los pacientes como para los sistemas sanitarios.

\textbf{ Acceso Desigual a la Atención Médica}

Existen notables disparidades en el acceso a los servicios médicos que pueden influir significativamente en la detección y tratamiento oportunos de la RD. Estas variaciones son particularmente evidentes entre distintas regiones y estratos socioeconómicos, y requieren ser abordadas en las políticas de salud para garantizar un tratamiento equitativo.

\textbf{ Impacto Psicológico y Comunitario}

El deterioro visual grave resultante de la RD afecta no solo a nivel individual, sino que también repercute en el entorno familiar y comunitario del paciente, exacerbando problemas psicosociales como el estrés y la depresión.

\textbf{ Costos Económicos a Nivel Macro y Micro}

Además de los gastos médicos directos, la RD acarrea costos indirectos por la pérdida de productividad laboral. Analizar estos aspectos desde una perspectiva global y local ofrece un panorama más claro para diseñar intervenciones efectivas y contextualizadas.

\textbf {Innovaciones Tecnológicas en Detección y Tratamiento}

La revolución tecnológica en la salud ha introducido herramientas como la inteligencia artificial y la telemedicina, las cuales están cambiando la forma de diagnóstico y tratamiento de la RD. Estas innovaciones presentan nuevas oportunidades pero también desafíos, particularmente en términos de acceso equitativo.

\textbf {Políticas Públicas y Estrategias de Prevención}
Es esencial evaluar críticamente las políticas y estrategias de salud pública vigentes, identificando áreas de mejora para fortalecer la prevención y el tratamiento de la RD, a través de campañas más efectivas y programas de detección mejorados.

\textbf {Futuro y Sostenibilidad del Sistema de Salud}

Ante el incremento esperado en la prevalencia de la diabetes, es imperativo que los sistemas de salud se preparen para manejar esta carga creciente de manera eficiente y sostenible.

\subsection{Problema General}
\newcommand{\ProblemaGeneral}{
	¿De qué manera la detección inadecuada contribuye a la incidencia de ceguera y deterioro visual severo en la población diabética específicamente a través de la retinopatía diabética?
}
\ProblemaGeneral
\subsection{Problemas Espec\'{i}ficos}
\newcommand{\Pbone}{
	¿Hasta qué punto la precisión de los modelos de deep learning en la detección temprana de la retinopatía diabética afecta la tasa de diagnósticos correctos y oportunos en comparación con los métodos convencionales de cribado?
}
\newcommand{\Pbtwo}{
	¿De qué forma la calidad y disponibilidad de los datos de entrenamiento impactan la capacidad de los modelos de deep learning para predecir de manera efectiva la retinopatía diabética, y cuál es su influencia en la prevención de la ceguera en pacientes diabéticos?
}
\newcommand{\Pbthree}{
	¿Cómo incide la variabilidad intra e inter observador en la anotación de datos en la precisión de los modelos de deep learning para detectar la retinopatía diabética en diversas poblaciones?
}
\newcommand{\Pbfour}{
	W
}
\newcommand{\Pbfive}{
	ES
}

\begin{itemize}
	\item \Pbone
	\item \Pbtwo
	\item \Pbthree
\end{itemize}

\section{Objetivos de la Investigación}

\subsection{Objetivo General}
\newcommand{\ObjetivoGeneral}{
	Evaluar la precisión, fiabilidad y aplicabilidad de los modelos de deep learning en la identificación temprana de la retinopatía diabética, para establecer una metodología que contribuya a una detección más efectiva de la enfermedad. }
}
\ObjetivoGeneral
\subsection{Objetivos Espec\'{i}ficos}
\newcommand{\Objone}{
	Evaluar y comparar la precisión de modelos seleccionados de deep learning en la identificación de características tempranas de la retinopatía diabética, frente a la precisión de los métodos de cribado tradicionales, y determinar su impacto en la mejora de los diagnósticos oportunos.
}
\newcommand{\Objtwo}{
	Investigar el efecto de la calidad y la disponibilidad de datos en la precisión de los modelos de deep learning y su capacidad para detectar la retinopatía diabética, y cómo esto podría contribuir a la prevención de la ceguera en la población diabética.
}
\newcommand{\Objthree}{
	Analizar el grado de variabilidad en la anotación de datos por diferentes observadores y su influencia en la eficacia de los modelos de deep learning, con el objetivo de identificar y proponer estrategias para mejorar la consistencia en la detección de la retinopatía diabética en poblaciones variadas.
}
\newcommand{\Objfour}{
	DSAD
}
\newcommand{\Objfive}{

}	

\begin{itemize}
	\item {\Objone}
	\item {\Objtwo}
	\item {\Objthree}


\end{itemize}

\section{Justificación de la Investigación}

\subsection{Teórica}
Este estudio se centra en la aplicación de modelos de aprendizaje profundo para la detección temprana de la retinopatía diabética, un problema de salud prevalente entre los diabéticos en Perú. Investigaciones recientes indican que hasta un 15.1\% de los pacientes en programas específicos de diabetes presentan esta condición, con una mayoría sufriendo de formas no proliferativas. Este proyecto busca profundizar el entendimiento de cómo los avances en inteligencia artificial pueden mejorar significativamente la detección y el seguimiento precoz de esta afección. Al hacerlo, la investigación aportará valiosos conocimientos sobre las capacidades y restricciones de las tecnologías emergentes en la oftalmología, enriqueciendo la literatura académica tanto a nivel local como internacional.

\subsection{Práctica}
Desde un punto de vista práctico, este trabajo de investigación tiene el potencial de generar una mejora considerable en  el método de pre-detección de la retinopatía diabética en Perú, donde comúnmente el diagnóstico ocurre en fases muy avanzadas. Implementar modelos de deep learning para la identificación temprana de la enfermedad podría facilitar intervenciones preventivas más efectivas, aliviar la carga económica sobre el sistema de salud y mejorar significativamente los resultados para los pacientes. Integrar esta tecnología en los sistemas públicos de salud mejoraría la accesibilidad y eficacia del diagnóstico de la RD, especialmente en zonas donde los recursos son escasos.

\subsection{Metodológica}. 
Metodológicamente, este estudio se distingue por su análisis exhaustivo y comparativo de diversos modelos de deep learning en un entorno clínico real. La metodología rigurosa que se aplicará no solo evaluará la precisión de estos modelos, sino que también explorará adaptaciones necesarias para maximizar su eficacia en el contexto específico de Perú. Esto proporcionará una base sólida para futuras investigaciones y aplicaciones de IA en el tratamiento de otras condiciones médicas, además de influir en el desarrollo de políticas de salud pública relacionadas con la implementación de nuevas tecnologías.

\section{Delimitación del Estudio}

\subsection{Espacial}
Este estudio se delimita al análisis de datos obtenidos de bases de datos internacionales reconocidas, específicamente el Messidor dataset y el APTOS 2019. Estas bases contienen imágenes de fondo de ojo de pacientes diabéticos, recolectadas bajo diversos estudios clínicos. La selección de estas bases de datos se debe a su amplio uso en la investigación académica y su relevancia para validar la precisión de modelos de deep learning en el contexto de la retinopatía diabética. La investigación no involucrará la recolección de nuevos datos clínicos ni se realizarán pruebas directas con pacientes en Perú o cualquier otra región, concentrándose exclusivamente en el análisis técnico y comparativo de los datos ya existentes.

\subsection{Temporal}
La investigación se llevará a cabo durante el año académico 2024, comenzando en enero y concluyendo en diciembre del mismo año. Este marco temporal ha sido seleccionado para alinear el estudio con el calendario académico y permitir un tiempo adecuado para la planificación, ejecución y análisis de la evaluación de los modelos. Durante este período, se realizará la selección de modelos, el procesamiento de datos, la ejecución de pruebas computacionales y la análisis de los resultados.

\subsection{Conceptual}
La investigación está enfocada en la evaluación técnica de modelos de deep learning específicos que han sido previamente desarrollados y aplicados en la detección de la retinopatía diabética. La delimitación conceptual abarca la validación de la efectividad de estos modelos en términos de precisión, sensibilidad, especificidad y otras métricas relevantes para el diagnóstico automatizado a través de imágenes. Se excluyen del estudio la creación de nuevos modelos de IA, cualquier intervención médica directa con pacientes, y la exploración de tratamientos para la retinopatía. Este enfoque permite una concentración rigurosa en la evaluación del rendimiento de tecnologías específicas en un contexto controlado y basado en datos, proporcionando una evaluación crítica de su utilidad práctica y limitaciones.

\section{Hipótesis}

\subsection{Hipótesis General}
\newcommand{\HipotesisGeneral}{
	La precisión de los modelos de deep learning preexistentes en la detección temprana de la retinopatía diabética es significativamente alta, lo que sugiere su viabilidad como herramientas eficientes para la identificación preliminar de esta condición en las poblaciones examinadas a través de las bases de datos Messidor y APTOS 2019
}
\HipotesisGeneral
\subsection{Hipótesis Específicas}
\newcommand{\Hone}{
	 Los modelos de deep learning seleccionados demostrarán una precisión significativamente superior en la detección de signos tempranos de retinopatía diabética en imágenes retinianas en comparación con los métodos de cribado estándar, lo que se traduce en una reducción de los diagnósticos tardíos de la enfermedad.
}
\newcommand{\Htwo}{
	 La calidad y la disponibilidad de los datos utilizados para entrenar y probar los modelos de deep learning tendrán un impacto directo y positivo en la precisión de la detección de la retinopatía diabética, lo que potencialmente podría disminuir la incidencia de ceguera entre los pacientes diabéticos.
}
\newcommand{\Hthree}{
	La variabilidad en la anotación de datos entre diferentes observadores afectará significativamente el rendimiento de los modelos de deep learning, y la implementación de protocolos estandarizados de anotación mejorará la consistencia y exactitud en la detección de la retinopatía diabética a través de diversas poblaciones.
}
\newcommand{\Hfour}{
	cv
}
\newcommand{\Hfive}{
	xws
}
\begin{itemize}
	\item \Hone
	\item \Htwo
	\item \Hthree

\end{itemize}

\subsection{Matriz de Consistencia}
A continuación se presenta la matriz de consistencia elaborada para la presente investigación (véase Anexo \ref{1:table}).
